%!TEX encoding = UTF-8 Unicode
%!TEX root = ../livre-can.tex


\chapter{Calcul de la fréquence}

%--- Pour supprimer tout en-tête et pied de page sur la 1re page d'un chapitre
\thispagestyle{empty}

Peut-on toujours atteindre les fréquences de bus désirées avec les cartes dont on dispose ?

Par exemple, l'auteur a été confronté au problème suivant : il voulait réaliser un réseau CAN entre un micro-contrôleur LPC2294 de NXP, conduit par un quartz de fréquence $14,7456$ MHz, et des PIC18F26K80 de Microchip, conduits par des quartz à $16$ MHz.

Le micro-contrôleur LPC2294 est équipé d'un multiplieur de fréquence programmable, la fréquence obtenue ne devant pas dépasser $60$ MHz. En programmant le facteur $4$, on aboutit à $59,9824$ MHz. Cette horloge sert de base de temps pour les modules CAN intégrés. 

Le PIC18F26K80 dispose aussi d'un multiplieur de fréquence optionnel, non programmable, d'un facteur fixe égal à $4$ : l'horloge du module CAN intégré a donc une fréquence de $64$ MHz.

Peut-on réaliser un bus à $1$ Mbit/s ? Cela ne pose pas de problème particulier pour les PIC18F26K80, mais, par une division entière de $58,9824$ MHz, on ne peut jamais atteindre exactement $1$ Mbit/s : en divisant par $60$, on obtient $983$ kbit/s, soit un écart de $1,7$ \% avec $1$ Mbit/s. En appliquant les résultats du \refChapterTitlePage{chapitreCalculBit}, on peut constater que c'est très limite \pasFini.

Les questions que l'on se pose alors :
\begin{itemize}
  \item existe-t'il pour le LPC2294 un réglage qui permette de mieux s'approcher de la fréquence de $1$ Mbit/s, et qui soit acceptable ?
  \item sinon, quelle fréquence choisir, qui puisse être commune aux PICs et au LPC2294 ?
\end{itemize}

Ce sont les réponses à ces deux questions qui vont être apportées dans ce chapitre.

Nous allons présenter des codes écrits en C++, basés sur une recherche exhaustive de toutes les réglages possibles des modules CAN d'un micro-contrôleur. Avec la vitesse actuelle des ordinateurs de bureau, il est inutile d'adopter des algorithmes efficaces : avec une recherche en force brute, les réponses sont quasi instantanées.





% \footnote{\url{http://www.nxp.com/products/microcontrollers-and-processors/arm-processors/lpc-arm7-arm9-mcus/lpc-arm7-microcontrollers/lpc2100-200-300-400/16-32-bit-arm-microcontrollers-256-kb-isp-iap-flash-with-can-10-bit-adc-and-external-memory-interface:LPC2294FBD144}}